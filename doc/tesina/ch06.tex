
\section{Conclusiones}

	Nos planteamos el objetivo de realizar una traducción para permitir ejecutar modelos PowerDEVS dentro de la herramienta de simulación QSS-Solver, 
	para lo cual se generaron de forma manual\footnote{La creación de estos modelos requiere conocimientos de PowerDEVS y de Modelica} los modelos 
	Modelica equivalentes a los bloques PowerDEVS utilizados en los ejemplos del Capitulo 5\footnote{que se encuentran disponibles en el Apendice A},
        se desarrollo un algoritmo de traducción para la estructura de archivos PDS de PowerDEVS, el cual incluye la traducción de estructuras planas,
	 es decir sin jerarquía, y la traducción de una estructuras jerárquica a estructuras plana, la cual nos permite encarar cualquier estructura 
	del archivo PDS, mientras se cuenten con los modelos Modelica generados manualmente. Además se implementaron transformaciones automáticas 
	sobre el código Modelica generado para soportar construcciones que no habían sido implementadas en QSS-Solver, condicional \texttt{if...then}, 
	producto escalar y arreglos bidimensionales.

	Se mostró, en detalles, la implementación y ejemplos aplicados junto con las mejoras en el tiempo de simulación, las cuales alcanzan cerca al 90\%
	 en modelos grandes, mostramos además como se mantiene los valores obtenidos de la simulación a través de comparar las gráficas de los resultados obtenidos.

\section{Trabajos Futuros}
	Para poder convertir una mayor variedad de modelos, de PowerDEVS, es necesario escribir más modelos en Modelica equivalentes a los modelos atómicos PowerDEVS.
	Adicionalmente, algunos modelo atómicos necesitan expandir el QSS-Solver para poder ser ejecutados, el caso más representativo es el de los modelos de tiempo
	 real lo cual requier modificar el QSS-Solver para soportarlos.

	Durante el desarrollo de este trabajo, la librería modelicacc \footnote{http://sourceforge.net/projects/modelicacc/?source=directory}, ha sido actualizada 
	con un nuevo parser, el cual deberá ser actualizado en este trabajo.

	La expansión de variables vectoriales, donde los valores provienen de expresiones del entorno Scilab, es siempre tratado como un valor escalar, y no como 
	uno vectorial, para determinar cómo se lo debería tratar es necesario realizar un análisis de los tipos de las variables expandidas. 
	Lo cual podría mejorar los mensajes de error cuando se genera el modelo final.

	Además es deseable que podamos llamar a esta conversión desde el mismo PowerDEVS y ejecutar el modelo resultante en el QSS-Solver.

\todo[inline]{algunas cosas que me quedaron en el tintero: 
    Hay que mencionar en la sección 3.2 cómo describís un modelo continuo y cómo uno con discontinuidades. no hay mención de nada.
    Siguiendo eso, mencioná en esa sección que en el apéndice A están los modelos desarrollados en esta tesina.
    
    Falta seguro la parte de equivalencia que habíamos hablado. Hay que poner de alguna forma que los modelos resultantes son equivalentes en la simualción por construcción. 
}
