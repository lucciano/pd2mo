\documentclass[a4paper,	11pt]{article}
%-----------Paquetes-------------------------

\usepackage[utf8]{inputenc}
\usepackage{amsmath}
\usepackage[spanish]{babel}
%\renewcommand{\abstractname}{Resumen: }

\begin{document}

\renewcommand\floatpagefraction{.9}
\renewcommand\topfraction{.9}
\renewcommand\bottomfraction{.9}
\renewcommand\textfraction{.1}
\setcounter{totalnumber}{50}
\setcounter{topnumber}{50}
\setcounter{bottomnumber}{50}

\title{Conversión de modelos PowerDEVS al lenguaje Modelica}
\author{Tesinista: Luciano Andrade \\ Director: Federico Bergero, Co-Director: Ernesto Kofman} 

\maketitle
\section{Resumen}
En este trabajo se describe la implementación de una aplicación para convertir modelos descriptos en la herramienta PowerDEVS a modelos en el lenguaje Modelica, más especificamente en $\mu$Modelica, con el fin de aprobechar la velocidad de simulación del 'QSS Solver', permitiendo describir las simulaciones en el entorno PowerDEVS y ejecutando las simulaciones en 'QSS Solver'


\section{Introducción}
\subsection{Motivación y Objetivos}
PowerDEVS es una herramienta de simulación de sistemas híbridos, basado en el formalismo DEVS, con una interfaz grafica orientada a bloques, donde los bloques pueden ser conectados entre si, modificado los parametros, posibilidad de conectarse con el entorno Scilab para poder utilizar expresiones y herramientas de cálculo provistas por este entorno.

La resoluci ́on de ecuaciones diferenciales ordinarias, requiere el uso de m ́etodos de integraci ́on num ́erica. Todos los algoritmos tradicionales de integraci ́on se basan en la discretizaci ́on de la variable independiente (que generalmente representa el tiempo). Las rutinas que implementan estos algoritmos, se denominan solvers y existen gran variedad de implementaciones de los mismos en diferentes lenguajes de programaci ́on. Los M ́etodos de Integraci ́on Num ́erica QSS (Quantized State System), a diferencia de los m ́etodos de integración tradicionales, realizan la discretizaci ́on sobre las variables de estado. En consecuencia, convierten los sistemas continuos en sistemas de eventos discretos, y tienen grandes ventajas para simular sistemas con discontinuidades.
Si bien PowerDEVS, implementa la totalidad de los algoritmos de QSS, resultan ineficientes, dado que malgastan gran parte de la carga computacional en la transmisión de eventos entre submodelos.

Para solventar este hecho se desarrollo una familia de QSS stand-solver, el cual requiere un modelo descripto en lenguaje C el cual contiene las equaciones difereciales, las funciones de cruce de cero asi como la información estructural requerida por los algoritmos QSS. Estos solvers obtienen una mejora de performace de hasta un orden de maginitud comparado con otras implementaciones DEVS.
Sobre este se desarrollo una herramineta la cual genera a partir de un modelo µ-Modedelica (un subconjunto del lenguaje Modelica) el modelo requerido para el QSS solver.

Con el objetivo de utilizar los mejoras de velocidad y mantener un entorno amigable con el usuario, se creo una herramienta capas de convertir un modelo PowerDEVS en un modelo µ-Modelica.


\subsection{Trabajo relacionado}
¿Cuales?
\subsection{Alcance}
 Diferentes limitaciones entre Devs y Modelica y la transformación, ademas de la diferencia entre la composicion de los modelos.
 diferencia superficiales en los formalismos

\section{Conceptos Previos}
\subsection{Modelado y Simulación}
\subsubsection{Sistemas Continuos y Discretos}
\subsubsection{Métodos de Integración numérica}
\subsection{Formalismo DEVS}
\subsubsection{Atómicos }
	descripcion
\subsubsection{Acoplados}
	descripcion, el aplanado se ve más adelante
\subsection {Métodos de integración de QSS}
\subsection {PowerDEVS}
\subsubsection{Modelos Vectoriales}
\subsection{Modelica}
\subsection{QSS Stand Alone Solver}
\subsubsection{$\mu$Modelica}

\section{Conversión de modelos DEVS}
\subsection{Modelos Atómicos}
Cómo se traducen (es conocimiento del modelador, no automático)
\subsection{Modelos Vectoriales}
Consideraciones  y anotaciones

\subsection{Modelos Acoplados Planos}
Modelos acoplados solo con modelos atómicos adentro.

Mencionar el algoritmo (traducción de conexiones y ``aplanado'' de cada uno de los atómicos hijos``)	

\subsection{Equivalencia semántica de la conversión}

\section{Modelos Acoplados Jerárquicos}
Explicar cuándo se utilzan y que resolvemos el problema aplanando los acoplados

\subsection{Algoritmo de aplanado}
Describir el algoritmo para PDS


\subsection{Comparación de performance}
¿No deberia estar despues de "Ejemplos de Aplicación"?

\section{Detalles de Implementación}
API Powerdevs, AST Modelica
Traverser
Modelica Transformer


\section{Ejemplos de Aplicación}
tamaños de las vectores y comparativas de 
\subsection{Vector/airs}
\subsection{Vector/lcline}

\section{Conclusiones y Trabajo a futuro}

\bibliographystyle{plain}
\begin{small}
\bibliography{tesina_luciano}
\end{small}
\end{document}
