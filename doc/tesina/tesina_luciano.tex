  \documentclass[a4paper,	11pt]{article}
%-----------Paquetes-------------------------

\usepackage[utf8]{inputenc}
\usepackage{amsmath}
\usepackage[spanish]{babel}
%\renewcommand{\abstractname}{Resumen: }

\begin{document}

\renewcommand\floatpagefraction{.9}
\renewcommand\topfraction{.9}
\renewcommand\bottomfraction{.9}
\renewcommand\textfraction{.1}
\setcounter{totalnumber}{50}
\setcounter{topnumber}{50}
\setcounter{bottomnumber}{50}

\title{Conversión de modelos PowerDEVS al lenguaje Modelica}
\author{Tesinista: Luciano Andrade \\ Director: Federico Bergero, Co-Director: Ernesto Kofman} 

\maketitle
\section{Resumen}
En este trabajo se descrive la implementación de una aplicación capaz de convertir modelos descriptos en la herramienta PowerDEVS a modelos en el Modelica, más especificamente en $\mu$Modelica, con el fin de aprobechar la velocidad de simulación del 'QSS Solver' y aprobechar su velocidad de simulación, permitiendo describir las simulaciones en el entorno PowerDEVS.


\section{Introducción}
\subsection{Motivación y Objetivos}
¿Que es un modelado de un sistemas?
¿Que es una simulación?

%bibliografia sobre devs

\subsection{Trabajo relacionado}
¿Cuales hay ?
\subsection{Alcance}
 Diferentes limitaciones entre Devs y Modelica y la transformación, ademas de la diferencia entre la composicion de los modelos.
 diferencia en los formalismos

\section{Conceptos Previos}
\subsection{Modelado y Simulación}
\subsubsection{Sistemas Continuos y Discretos}
\subsubsection{Métodos de Integración numérica}
\subsection{Formalismo DEVS}
\subsubsection{Atómicos }
\subsubsection{Acoplados}
Equivalencia entre atómicos y acoplados.

\subsection {Métodos de integración de QSS}
\subsection {PowerDEVS}
\subsubsection{Modelos Vectoriales}
\subsection{Modelica}
\subsection{QSS Stand Alone Solver}
\subsubsection{$\mu$Modelica}

\section{Conversión de modelos DEVS}
\subsection{Modelos Atómicos}
Cómo se traducen (es conocimiento del modelador, no automático)
\subsection{Modelos Vectoriales}
?

\subsection{Modelos Acoplados Planos}
Modelos acoplados solo con modelos atómicos adentro.

Mencionar el algoritmo (traducción de conexiones y ``aplanado'' de cada uno de los atómicos hijos``)	

\subsection{Equivalencia semántica de la conversión}

\section{Modelos Acoplados Jerárquicos}
Explicar cuándo se utilzan y que resolvemos el problema aplanando los acoplados

\subsection{Algoritmo de aplanado}
Describir el algoritmo para PDS


\subsection{Comparación de performance}

\section{Detalles de Implementación}
API Powerdevs, AST Modelica
Traverser
Modelica Transformer


\section{Ejemplos de Aplicación}
tamaños de las vectores y comparativas de 
\subsection{Vector/airs}
\subsection{Vector/lcline}

\section{Conclusiones y Trabajo a futuro}

\bibliographystyle{plain}
\begin{small}
\bibliography{tesina_luciano}
\end{small}
\end{document}
