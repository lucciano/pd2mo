  \documentclass[a4paper,	11pt]{article}
%-----------Paquetes-------------------------

\usepackage[utf8]{inputenc}
\usepackage{amsmath}
\usepackage[spanish]{babel}
%\renewcommand{\abstractname}{Resumen: }

\begin{document}

\renewcommand\floatpagefraction{.9}
\renewcommand\topfraction{.9}
\renewcommand\bottomfraction{.9}
\renewcommand\textfraction{.1}
\setcounter{totalnumber}{50}
\setcounter{topnumber}{50}
\setcounter{bottomnumber}{50}

\title{Conversión de modelos PowerDEVS al lenguaje Modelica}
\author{Tesinista: Luciano Andrade \\ Director: Federico Bergero, Co-Director: Ernesto Kofman} 

\maketitle
\section{Resumen}
En este trabajo se descrive la implementación de una aplicación capaz de convertir modelos descriptos en la herramienta PowerDEVS a modelos en el Modelica, más especificamente en $\mu$Modelica, con el fin de aprobechar la velocidad de simulación del 'QSS Solver' y aprobechar su velocidad de simulación, permitiendo describir las simulaciones en el entorno PowerDEVS.


\section{Introducción}
\subsection{Motivación y Objetivos}
\subsection{Trabajo relacionado}
\subsection{Alcance}

\section{Conceptos Previos}
\subsection{Modelado y Simulación}
\subsubsection{Sistemas Continuos y Discretos}
\subsubsection{Métodos de Integración numérica}
\subsection{Formalismo DEVS}
\subsubsection{Atómicos }
\subsubsection{Acoplados}
Equivalencia entre atómicos y acoplados.

\subsection {Métodos de integración de QSS}
\subsection {PowerDEVS}
\subsubsection{Modelos Vectoriales}
\subsection{Modelica}
\subsection{QSS Stand Alone Solver}
\subsubsection{$\mu$Modelica}

\section{Conversión de modelos DEVS}
\subsection{Modelos Atómicos}
Cómo se traducen (es conocimiento del modelador, no automático)
\subsection{Modelos Vectoriales}
?

\subsection{Modelos Acoplados Planos}
Modelos acoplados solo con modelos atómicos adentro.

Mencionar el algoritmo (traducción de conexiones y ``aplanado'' de cada uno de los atómicos hijos``)	

\subsection{Equivalencia semántica de la conversión}

\section{Modelos Acoplados Jerárquicos}
Explicar cuándo se utilzan y que resolvemos el problema aplanando los acoplados

\subsection{Algoritmo de aplanado}
Describir el algoritmo para PDS

\subsection{Comparación de performance}

\section{Detalles de Implementación}

\section{Ejemplos de Aplicación}
\subsection{Vector/airs}
\subsection{Vector/lcline}

\section{Conclusiones y Trabajo a futuro}

\newpage


\section{Marco del proyecto}
El trabajo se enmarca en el proyecto PID-ING386, Modelado, Simulación y Control en Tiempo Real con
Aplicaciones en Electrónica de Potencia, UNR y PICT 2012-0077 de la Agencia de Nacional de Promoción Científica y Tecnología.  

La tesina será dirigida por el Dr. Federico Bergero y el Dr. Ernesto Kofman. La tesina no es remunerada.

\section{Estado del Arte}
 Comentaremos aquí brevemente, tanto en qué consiste la herramienta PowerDEVS, el lenguaje Modelica y trabajos relacionados con el presente plan.
\subsection*{PowerDEVS}
         \label{powerdevs}
        
         PowerDEVS\cite{BK11} es un entorno de simulación de
         sistemas de eventos discretos (DEVS) de propósito general. Fue desarrollado
         en la Facultad de Ciencias Exactas, Ingeniería
         y Agrimensura de la Universidad Nacional de Rosario como proyecto
         final y es mantenido por nuestro grupo de investigación.
         
         Esta herramienta posee una interfaz gráfica por la cual permite describir modelos
          DEVS de una forma gráfica y sencilla. La interfaz gráfica consiste esencialmente
          en un editor de diagramas de bloques con una funcionalidad
          muy similar a la de Simulink donde pueden conectarse los distintos bloques atómicos. 
          
          PowerDEVS incluye una librería de bloques DEVS predefinidos que pueden ser utilizado
	  para crear modelos más complejos. La herramienta permite también agregar bloques
	  especificando su funcionalidad a través de una descripción C++. Dentro de esta librería	
	  existen bloques integradores que utilizan métodos de quantificación de estado \cite{Cel06}
	  para aproximar sistemas continuos a través de eventos discretos.
	  	  
 
\subsection*{Modelica}
         \label{modelica}
        Modelica \cite{Fri98} es un lenguaje orientado a objetos, para el modelado de sistemas complejos, con componentes, 
mecánicos, eléctricos, electrónicos, hidráulicos, térmicos, etc. El lenguaje no tiene restricciones de uso (licencia Modelica V2)
y es desarrollado por la asociación sin fines de lucro Modelica Asociation. Un modelo Modelica es una representación
textual aunque las herramientas de modelado y simulación de Modelica permiten componer el modelo gráficamente sin 
tener que escribirlo. 

\subsection*{Modelica y DEVS}
Es interesante establecer la relación entre estos dos lenguajes de modelado. 
Los sistemas de eventos discretos descriptos en DEVS pueden ser representado sin pérdida de generalidad en Modelica.
Modelica puede representar también sistemas continuos los cuales, como antes mencionamos, pueden ser aproximados en DEVS con los métodos de QSS.

La transformación de un modelo Modelica en un modelo DEVS equivalente fue estudiada previamente por nuestro grupo de investigación en \cite{Flo11}. 
En el trabajo citado se obtuvo una ganancia en performance  al realizar la simulación en PowerDEVS con los métodos de QSS.


\section{Motivación y Objetivos}
La herramienta de modelado y simulación  PowerDEVS \cite{BK11} permite describir sistemas en el formalismo de eventos discretos DEVS y luego realizar
la simulación. Aunque los sistemas de tiempo continuo no pueden ser representados directamente en el formalismo DEVS, éstos pueden 
ser aproximados por los métodos de cuantificación de estados QSS \cite{Cel06}. La implementación de estos métodos en una herramienta de modelado y simulación
DEVS es trivial pero resulta ineficiente. Esto motivó el desarrollo de una implementación autónoma \cite{Fer12} (fuera del formalismo DEVS) la cual resulta un orden
de magnitud más eficiente en tiempo de simulación. Esta herramienta acepta modelos descriptos en un subconjunto del lenguaje Modelica \cite{Fri98}, por lo tanto
los modelos realizados en PowerDEVS no pueden ser simulados directamente en ella sino que deben ser traducidos a Modelica.

El trabajo a grandes rasgos consiste en el estudio y desarrollo de una herramienta para la conversión automática de modelos
descriptos en PowerDEVS hacia un subconjunto del lenguaje Modelica. 
Para ello habrá que desarrollar modelos Modelica para cada uno de los bloques de la librería de PowerDEVS, y luego generar un modelo Modelica
que represente la estructura del modelo DEVS.


\section{Tareas a realizar y Cronograma}
La tesina tendrá una duración de seis meses con una carga
horaria de 4 hs diarias. Tendrá inicio el 01/03/2014 y finalización el
31/09/2014.

En forma global, el trabajo consiste en la implementación de un traductor de modelos descriptos en PowerDEVS al lenguaje
Modelica. Dependiendo del resultado obtenido y de la disponibilidad de tiempo podremos integrarlo a la herramienta PowerDEVS
de forma de poder simular los modelos con el simulador autónomo antes presentado.
% 
Las tareas a realizar son: 
\begin{itemize}
	\item Estudio del lenguaje Modelica y de la herramienta PowerDEVS (1 semana). 
	\item Como ejercicio de prueba, desarrollar la traducción de forma manual para encontrar posibles problemas.(1 semana).
	\item Desarrollar modelos Modelica equivalente para cada bloque (o la mayoría) de la librería de PowerDEVS (2 semanas)
	\item Estudiar, analizar e implementar el algoritmo para la conversión (1 mes).
	\item Integrar la solución con la herramienta PowerDEVS para facilitar la tarea del usuario (1 semana).
	\item Depurar, generar casos de prueba, documentar y anotar el código (1 mes).
	\item Realizar el informe de tesina (1 mes).
\end{itemize}



\section{Factibilidad y Antecedentes}
La tesina será realizada en el grupo de Simulación y Control de Sistemas Dinámicos del instituto CIFASIS. Los directores cuentan 
con experiencia en el desarrollo de software en particular en el área en cuestión (métodos de integración numérica, 
herramientas de simulación y compiladores) y con recursos (computadoras, bibliografía, acceso a revistas) para completar el proyecto
en el tiempo estipulado.

\bibliographystyle{plain}
\begin{small}
\bibliography{tesina_luciano}
\end{small}
\end{document}
