\documentclass[a4paper,	11pt]{report}
%-----------Paquetes-------------------------

\usepackage[utf8]{inputenc}
\usepackage{amsmath}
\usepackage[spanish]{babel}
%\renewcommand{\abstractname}{Resumen: }

\begin{document}

\renewcommand\floatpagefraction{.9}
\renewcommand\topfraction{.9}
\renewcommand\bottomfraction{.9}
\renewcommand\textfraction{.1}
\setcounter{totalnumber}{50}
\setcounter{topnumber}{50}
\setcounter{bottomnumber}{50}

\title{Conversión de modelos PowerDEVS al lenguaje Modelica}
\author{Tesinista: Luciano Andrade \\ Director: Federico Bergero, Co-Director: Ernesto Kofman} 

\maketitle
\tableofcontents

\begin{abstract}
%\section{Resumen}
En este trabajo se describe la implementación de una aplicación para convertir modelos descriptos en la herramienta PowerDEVS a modelos en el lenguaje Modelica, más especificamente en $\mu$Modelica, con el fin de aprovechar la velocidad de simulación del 'QSS Solver', permitiendo describir las simulaciones en el entorno PowerDEVS y ejecutando las simulaciones en 'QSS Solver'
\end{abstract}


\chapter{Introducción}
\section{Motivación y Objetivos}
PowerDEVS es una herramienta de simulación de sistemas híbridos, basado en el formalismo DEVS, con una interfaz grafica orientada a bloques, donde los bloques pueden ser conectados entre si, modificado los parametros, posibilidad de conectarse con el entorno Scilab para poder utilizar expresiones y herramientas de cálculo provistas por este entorno.

La resolución de ecuaciones diferenciales ordinarias, requiere el uso de métodos de integración numérica. Todos los algoritmos tradicionales de integración se basan en la discretización de la variable independiente (que generalmente representa el tiempo). Las rutinas que implementan estos algoritmos, se denominan solvers y existen gran variedad de implementaciones de los mismos en diferentes lenguajes de programación. Los Métodos de Integración Numérica QSS (Quantized State System), a diferencia de los métodos de integración tradicionales, realizan la discretización sobre las variables de estado. En consecuencia, convierten los sistemas continuos en sistemas de eventos discretos, y tienen grandes ventajas para simular sistemas con discontinuidades.
Si bien PowerDEVS, implementa la totalidad de los algoritmos de QSS, resultan ineficientes, dado que malgastan gran parte de la carga computacional en la transmisión de eventos entre submodelos.

Para solventar este hecho se desarrollo una familia de QSS stand-solver, el cual requiere un modelo descripto en lenguaje C el cual contiene las equaciones difereciales, las funciones de cruce de cero asi como la información estructural requerida por los algoritmos QSS. Estos solvers obtienen una mejora de performace de hasta un orden de maginitud comparado con otras implementaciones DEVS.
Sobre este se desarrollo una herramineta la cual genera a partir de un modelo $\mu$-Modedelica (un subconjunto del lenguaje Modelica) el modelo requerido para el QSS solver.

Con el objetivo de utilizar los mejoras de velocidad y mantener un entorno amigable con el usuario, se creo una herramienta capas de convertir un modelo PowerDEVS en un modelo $\mu$-Modelica.


\section{Trabajo relacionado}
¿Cuales?
\section{Alcance}
DEVS, Discrete Event System Specification (Especificación de Sistemas de Eventos Discretos), es un formalismo modular y jerarquico para modelar y analizar sistemas que pueden ser de eventos de tiempo discreto mediante tablas de transición, y con estados continuos mediantes que pueden ser descriptos mediantes equaciones diferenciales.

En el formalismo clasico DEVS, los modelos atómicos capturan el comportamiento del sistema, mientras los modelos Acoplados describen la estructura del sistema.

En particular los modelos atómicos en PowerDEVS son descriptos en clases C++, mientras que la estructura se encuentra definida en archivos pds y pdm.

Modelica es un lenguaje de modelado, orientado a objetos, declarativo, para el modelado orientado a componentes de sistemas complejos.

Para poder realizar nuestro objetivo es necesario primero contar con un modelo  en modelica para cada atómico PowerDEVS que deseemos convertir. 

De esto se desprenden las siguientes limitaciones importantes:
\begin{itemize}
	\item La semántica de los modelos convertidos depende de los modelos equivalentes a los DEVS atómicos 
	\item Solamente podemos convertir modelos cuyos componentes atómicos estén o puedan ser convertidos a $\mu$modelica.
\end{itemize}
	
\chapter{Conceptos Previos}
En este capítulo introducimos algunos conceptos básicos, necesarios para poder comprender este trabajo.
\section{Modelado y Simulación}

Modelado y Simulación de un Sistema es el proceso por el cual se desarrolla un modelo, el cual es luego simulado, de forma de obtener datos sobre el sistema. El modelo debe conservar las principales características del sistema, pero al mismo tiempo ser significativamente más simple, de forma que al momento de simularlo sea más eficaz utilizar la simulación que el sistema en si.

\subsection{Sistemas Continuos y Discretos}
Se considera un sistema continuo si las variables de este son conocidas en cada instante de tiempo, mientras que se considera discreto si las variables son conocidas en instantes de tiempo determinados.

En general los sistemas en estudio serán continuos, pero deberemos utilizar sistemas discretos puesto que la simulación en computadora así lo requiere, puesto que la misma computadora es un sistema discreto.

\subsection{Métodos de Integración numérica}
%Ber12 pag. 9
Un sistema continuo puede ser descripto por un modelo en espacios de estados de la forma:

\begin{equation} \label{eq1}
x(t) = f (x(t), u(t))
\end{equation}

\section{Formalismo DEVS}
Ber12 pag. 20
\subsection{Atómicos }

\subsection{Acoplados}

\subsection{Modelos Vectoriales}
BK11
	
\section{Métodos de integración QSS}
QSS paper

\section{PowerDEVS}
Powerdevs Paper

\section{Modelica}
\section{QSS Stand Alone Solver}
QSSPaper

\subsection{$\mu$Modelica}

\chapter{Conversión de modelos DEVS}
\section{Modelos Atómicos}
Cómo se traducen (es conocimiento del modelador, no automático)
\section{Modelos Vectoriales}
Consideraciones  y anotaciones en modelica, modificaciones

\section{Modelos Acoplados Planos}
Modelos acoplados solo con modelos atómicos adentro.	

Mencionar el algoritmo (traducción de conexiones y ``aplanado'' de cada uno de los atómicos hijos``)	

\section{Equivalencia semántica de la conversión}

\section{Modelos Acoplados Jerárquicos}
Explicar cuándo se utilzan y que resolvemos el problema aplanando los acoplados

\subsection{Algoritmo de aplanado}
Describir el algoritmo para PDS

\section{Comparación de performance}
¿No deberia estar despues de "Ejemplos de Aplicación"?

\chapter{Detalles de Implementación}
API Powerdevs, AST Modelica
Traverser
Modelica Transformer


\section{Ejemplos de Aplicación}
tamaños de las vectores y comparativas de 
\subsection{Vector/airs}
\subsection{Vector/lcline}
\nocite{*}
\section{Conclusiones y Trabajo a futuro}


\bibliographystyle{plain}
\bibliography{tesina_luciano}

\end{document}
