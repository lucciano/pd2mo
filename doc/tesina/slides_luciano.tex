%\title{Conversión de modelos PowerDEVS al lenguaje Modelica} % The short title appears at the bottom of every slide, the full title is only on the title page
%
%\author{Luciano Andrade} % Your name
%\institute[UNR] % Your institution as it will appear on the bottom of every slide, may be shorthand to save space
%{
%Universidad Nacional de Rosario\\ % Your institution for the title page
%\medskip
%\textit{andrade.luciano@gmail.com} % Your email address
%}
%\date{\today} % Date, can be changed to a custom date
%
%\begin{document}
%
%\begin{frame}
%\titlepage % Print the title page as the first slide
%\end{frame}
%
%\section{Introducción}
%	\subsection{Resumen}
%	\subsection{Motivación}
%	\subsection{Esquema General}
%\section{Conceptos Previos}
%	\subsection{Modelado y Simulación}
%	\subsection{Métodos de Integración numérica}
%	\subsection{Un ejemplo}
%	\subsection{Modelica}
%	\subsection{Métodos de Integración QSS}
%	\subsection{µ-Modelica}
%	\subsection{Stand–Alone QSS solver}
%	\subsection{Formalismo DEVS}
%		\subsubsection{Modelos Atómicos}
%		\subsubsection{Modelos Acoplados}
%		\subsubsection{Modelos Vectoriales}
%	\subsection{PowerDEVS}
%\section{Conversión de modelos DEVS}
%	\subsection{Modelos DEVS}
%		\subsubsection{Archivos PDS}
%		\subsubsection{Protocolo}
%	\subsection{Modelos Acoplados Planos}
%	\subsection{Modelos Acoplados Jerárquicos }
%	\subsection{Modelos Vectoriales}
%	\subsection{Transformaciones Extras}
%\section{Resultados}
%	\subsection{LK}
%	\subsection{Líneas de Transmisión}
%	\subsection{Comparaciones}
%\section{Conclusiones}
%
%\end{document}

\documentclass{beamer}

\AtBeginSection[]
{
  \begin{frame}<beamer>
    \frametitle{Outline for section \thesection}
    \tableofcontents[currentsection]
  \end{frame}
}

\title{Conversión de modelos PowerDEVS al lenguaje Modelica} 

\author{Luciano Andrade} 
\institute[UNR] 
{
Universidad Nacional de Rosario\\
\medskip
\textit{andrade.luciano@gmail.com}
}
\date{\today}

\begin{document}

\begin{frame}
\titlepage % Print the title page as the first slide
\end{frame}

\section{Introducción}
	\subsection{Resumen}
	\subsection{Motivación}
	\subsection{Esquema General}
\section{Conceptos Previos}
	\subsection{Modelado y Simulación}
	\subsection{Métodos de Integración numérica}
	\subsection{Un ejemplo}
	\subsection{Modelica}
	\subsection{Métodos de Integración QSS}
	\subsection{µ-Modelica}
	\subsection{Stand–Alone QSS solver}
	\subsection{Formalismo DEVS}
		\subsubsection{Modelos Atómicos}
		\subsubsection{Modelos Acoplados}
		\subsubsection{Modelos Vectoriales}
	\subsection{PowerDEVS}
\section{Conversión de modelos DEVS}
	\subsection{Modelos DEVS}
		\subsubsection{Archivos PDS}
		\subsubsection{Protocolo}
	\subsection{Modelos Acoplados Planos}
	\subsection{Modelos Acoplados Jerárquicos }
	\subsection{Modelos Vectoriales}
	\subsection{Transformaciones Extras}
\section{Resultados}
	\subsection{LK}
	\subsection{Líneas de Transmisión}
	\subsection{Comparaciones}
\section{Conclusiones}


\end{document}

