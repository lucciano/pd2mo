\documentclass{beamer}
\usepackage[spanish]{babel}
\usepackage[utf8]{inputenc}


\AtBeginSection[]
{
  \begin{frame}<beamer>
    \frametitle{Outline}
    \tableofcontents[
  currentsection,
  sectionstyle=show/show,
  subsectionstyle=show/shaded/hide
]
  \end{frame}
}

\title{Conversión de modelos PowerDEVS al lenguaje Modelica} 

\author{Luciano Andrade} 
\institute[UNR] 
{
Universidad Nacional de Rosario\\
\medskip
\textit{andrade.luciano@gmail.com}
}
\date{\today}

\begin{document}

\begin{frame}
\titlepage % Print the title page as the first slide
\end{frame}

\section{Introducción}
	\subsection{Resumen}
	\subsection{Motivación}
	\subsection{Esquema General}
\section{Conceptos Previos}
	\subsection{Modelado y Simulación}
	\subsection{Métodos de Integración numérica}
	\subsection{Un ejemplo}
	\subsection{Modelica}
	\subsection{Métodos de Integración QSS}
	\subsection{$\mu$-Modelica}
	\subsection{Stand–Alone QSS solver}
	\subsection{Formalismo DEVS}
		\subsubsection{Modelos Atómicos}
		\subsubsection{Modelos Acoplados}
		\subsubsection{Modelos Vectoriales}
	\subsection{PowerDEVS}
\section{Conversión de modelos DEVS}
	\subsection{Modelos DEVS}
		\subsubsection{Archivos PDS}
		\subsubsection{Protocolo}
	\subsection{Modelos Acoplados Planos}
	\subsection{Modelos Acoplados Jerárquicos }
	\subsection{Modelos Vectoriales}
	\subsection{Transformaciones Extras}
\section{Resultados}
	\subsection{LK}
	\subsection{Líneas de Transmisión}
	\subsection{Comparaciones}
\section{Conclusiones}

\end{document}

