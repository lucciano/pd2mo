\documentclass{beamer}
%\documentclass[12pt,handout]{beamer}
\usepackage[spanish]{babel}
\usepackage[utf8]{inputenc}
\usepackage{minted}
\usepackage{tikz}
\usepackage{smartdiagram}
\usetikzlibrary{trees}
\usepackage{graphicx} % Allows including images
\usepackage{booktabs} % Allows the use of \toprule, \midrule and \bottomrule in tables



\tikzstyle{every node}=[draw=black,thick,anchor=west]
\tikzstyle{selected}=[draw=red,fill=red!30]
\tikzstyle{optional}=[dashed,fill=gray!50]

\newcommand\TBox[2][]{%
  \tikz\node[draw,ultra thick,align=left,#1] {#2};\hskip2pt}

\makeatletter
\newcommand*{\centerfloat}{%
  \parindent \z@
  \leftskip \z@ \@plus 1fil \@minus \textwidth
  \rightskip\leftskip
  \parfillskip \z@skip}
\makeatother

\newcommand{\quotes}[1]{``#1''}

\usepackage{biblatex}
\addbibresource{tesina_luciano.bib}
\nocite{*}
\newcommand{\customcite}[1]{\citetitle{#1}, \citeyear{#1}}

\mode<presentation> {

\usetheme{default}

%\setbeamertemplate{footline} % To remove the footer line in all slides uncomment this line
%\setbeamertemplate{footline}[page number] % To replace the footer line in all slides with a simple slide count uncomment this line

%\setbeamertemplate{navigation symbols}{} % To remove the navigation symbols from the bottom of all slides uncomment this line
}

\AtBeginSection[]
{
  \begin{frame}<beamer>
 %   \frametitle{\thesection}
    \tableofcontents[currentsection]
  \end{frame}
}



%----------------------------------------------------------------------------------------
%	TITLE PAGE
%----------------------------------------------------------------------------------------

\title{Conversión de modelos PowerDEVS al lenguaje Modelica} % The short title appears at the bottom of every slide, the full title is only on the title page

\author{Luciano Andrade} % Your name
\institute[UNR] % Your institution as it will appear on the bottom of every slide, may be shorthand to save space
{
Universidad Nacional de Rosario\\ % Your institution for the title page
\medskip
\textit{andrade.luciano@gmail.com} % Your email address
}
\date{\today} % Date, can be changed to a custom date

\begin{document}

\begin{frame}
\titlepage % Print the title page as the first slide
\end{frame}

\section{Introducción}
\subsection{Motivación}
\begin{frame}
\begin{itemize}
	\item<1-> El sistema físico no se encuentra construido. 
	
	\item<2-> El experimento puede ser peligroso. Se realizan simulaciones para determinar si el experimento real \quotes{explotara}.

	\item<3-> El costo del experimento es demasiado alto o las herramientas necesarias no se encuentran disponibles o son muy costosas.

	\item<4-> Los tiempos del sistema no son compatibles con los tiempos del experimentador, ya sea porque es demasiado rápido o porque es demasiado lento.

	\item<5-> Variables de control, de estado y/o del sistema pueden no ser accesibles. Las simulaciones también nos permite manipular el modelo en formas que no podríamos manipular el sistema real.

	\item<6-> Eliminación de perturbaciones. Lo que nos permite aislar efectos particulares, y puede conducir a mejores apreciaciones sobre el comportamiento general del sistema.

	\item<7-> Eliminación de efectos de segundo orden (como no linealidades de componentes del sistema). 
\end{itemize}
\end{frame}

\section{Conceptos Previos}
\subsection{Modelado y Simulación}
\subsubsection{Sistemas Continuos y Discretos}
\subsubsection{Métodos de Integración}


\end{document}
