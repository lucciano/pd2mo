	A continuación se presenta el pseudo código que implementa las transformaciones descriptas en este trabajo.
	El código completo puede encontrarse en \url{https://github.com/lucciano/pd2mo}, el cual utiliza dos librerías, Modelica C Compiler 
	\footnote{http://sourceforge.net/projects/modelicacc/} la cual nos permite manipular la estructura de los modelos y evaluar los parámetros,
	y librería de PowerDEVS \footnote{http://sourceforge.net/projects/powerdevs/} para leer los archivos PDS.


\section{Programa Principal}

El Programa principal esta en el archivo main.cpp, el cual es responsable de la interfaz con el usuario (linea de comando) y de lanzar la transformación de la simulación, así como establecer los archivos desde donde se lee y hacia donde se escriben la simulación de PowerDEVS y Modelica, respectivamente.

La transformación de la simulación se encuentra separada en distintos módulos, implementados como clases C++, los cuales funcionan como una linea aplicándose uno detrás del otro, ló cual se intenta
describir en la Figura \ref{fig:pipeline}.
\begin{figure}[H]
\centerfloat
\smartdiagramset{
uniform color list=gray!60!black for 6 items,
back arrow disabled=true,
}
\smartdiagram[flow diagram:horizontal]{Parseo,Aplanado,Transformación Principal,Arreglo Bidimensional,Construcciones If,Producto Interno}
\caption{Esquema de transformaciones aplicadas}
\label{fig:pipeline}
\end{figure}

En el Procedimiento \ref{proc:main} se puede ver la creación de un objeto \texttt{modelCoupled} el cual representa la estructura interna del archivo PDS, 
esta estructura es aplanada y el objeto resultante genera un archivo PDS aplanado, el cual es convertido por la transformación principal en un 
	\texttt{AST\_StoredDefinition}, el cual representa un AST (por sus siglas en ingles Abstract Sintax Tree) de un modelo Modelica, el cual por último es 
	modificado por las transformaciones \texttt{If}, \texttt{mda}, \texttt{prod} y escrito a un archivo de salida.


\begin{algorithm}[H]
\begin{algorithmic}[1]
\State modelCoupled *cm $\gets$ parsePDS(QString::fromStdString(src\_infile));
\State modelCoupled *qm $\gets$ flatter::flat(cm);
\State flatted = replace(src\_infile, ".pds", ".flatted.pds");
\State generateCode(qm, QString::fromStdString(flatted), false, true);
\State modelname = replace(src\_infile, ".pds", "");
\State outfile = replace(src\_infile, ".pds", ".mo");
\State oFlogfile = replace(src\_infile, ".pds", ".log");
\State Pd2Mo q $\gets$ Pd2Mo();
\State q.transform(flatted, modelname, \&outfile, \&oFlogfile);
\State AST\_StoredDefinition sd $\gets$ parseFile(src\_outfile,\&r);
\State mda *m $\gets$ new mda();
\State If *i $\gets$ new If();
\State outfile $\ll$ m$\rightarrow$\Call{visitClass} 
		{prod$\rightarrow$visitClass( i$\rightarrow$visitClass( 
			*sd$\rightarrow$models()$\rightarrow$begin()))} $\ll$ endl;

\end{algorithmic}
\caption{main(src\_infile)}\label{proc:main}
\end{algorithm}

\section{Aplanado de modelos acoplados} 
Este módulo se encuentra en una clase llamada \emph{flatter} e implementa el aplanado de los modelos acoplados descripto en la sección \ref{aplanado}. 
Trabaja con un objeto \texttt{modelCoupled}, el cual como ya mencionamos, cuenta con la estructura del archivo PDS, es decir una lista de modelos y una lista de conexiones.
 
\begin{algorithm}[H]
\begin{algorithmic}[1]
\For{ModeloHijo en Lista de Modelos}
  	\If{Tipo de ModeloHijo es COUPLED}
  		\For{ModeloHijo2 en Lista de ModeloHijo$\rightarrow$ModeloHijo}
  			 	\If{Tipo de ModeloHijo2 es ATOMIC}
  			 		\State Copiamos el ModeloHijo2 al ModeloResultado;
				\Else
  			 		\State Copiamos el aplanado de ModeloHijo2;
				\EndIf
				\State posicionModelo $\gets$ Posición del ModeloHijo en el Modelo
  			 	\For{Conexión Interna ic del Modelo}
					\State (x,u);(y,v) $\gets$ ic
  			 		\Comment{Si la conexión involucra un modelo \quotes{aun no procesado}}
					\If{x \textgreater posicionModelo}
						\State x $\gets$ x + posicionModelo
					\EndIf
					\If{y \textgreater posicionModelo}
						\State y $\gets$ y + posicionModelo
					\EndIf

  			 		\If{Si la conexión involucra como destino el modelo acoplado ModeloHijo}
  			 			\Comment{Se crea una nueva conexión (en ModeloResultado) entre los modelos agregado recientemente según la conexión del puerto de entrada del ModeloHijo y el origen de la conexión}
						\For{Conexión Externa Entrante eic del ModeloHijo}
							\State (0,u1);(x1,v1) $\gets$ eic
							\If{u1 $==$ v}
								\State icadd $\gets$ (x,u);(x1+posicionModelo,v1)
								\State Agregamos icadd a las conexión del Modelo
							\EndIf
						\EndFor
  			 			\State La conexión se marca para ser borrada;
					\EndIf
\algstore{flattercontext}
\end{algorithmic}
\caption{flatter::flat}
\end{algorithm}

\begin{algorithm}[H]
\begin{algorithmic}[1]
\algrestore{flattercontext}
  			 		\If{Si la conexión involucra como origen el modelo acoplado ModeloHijo}
  			 			\Comment{Se crea una nueva conexión (en ModeloResultado) entre los modelos agregado recientemente según la conexión del puerto de salida del ModeloHijo y el destino de la conexión}
						\For{Conexión Externa Saliente eoc del ModeloHijo}
							\State (x1,u1);(0,v1) $\gets$ eic
							\If{v1 $==$ u}
								\State icadd $\gets$ (x1+posicionModelo,u1);(y,v)
								\State Agregamos icadd a las conexión del Modelo
							\EndIf

						\EndFor

  			 			\State La conexión se marca para ser borrada;
					\EndIf
  			 		\If{Si la conexión fue marcada para ser borrada}
						\State Se borra la conexión
					\EndIf
				\EndFor
		\EndFor
  	\Else

  		\State Copiamos el nodo ModeloHijo al ModeloResultado 
  		\State Copiamos las conexiones del ModeloHijo y cualquier otro ModeloHijo que ya haya sido procesado

	\EndIf
\EndFor
\State \Return ModeloResultado
\end{algorithmic}
\caption{flatter::flat (cont.)}
\end{algorithm}

\section{Transformación Principal}
La clase \emph{Pd2Mo} implementa las principales partes de la transformación, la cual incluye abrir el archivo PDS, e invocar el aplanado, obtener los diferentes modelos Modelica que representan los modelos atómico, prevenir la colisión de nombres, crear el modelo final y realizar las conexiones entre los modelos.

\begin{algorithm}[H]
\begin{algorithmic}[1]
\State modelCoupled *model $\gets$ parsePDS(qfilename);
\State AST\_ClassList classList $\gets$ getAsClassList(model); 
\State int j $\gets$ 0\;
\For{class en classList}
 	\If{La clase esta traducida a $\mu$Modelica}

 		\State Prefijamos las variables con el nombre del modelo $class$ y la posición $j$ que ocupan en la lista;
 		\State Remplazamos la entrada $class$ dentro de la lista por su copia producida en el paso anterior;
 	\EndIf
\EndFor
\State Creamos un modelo $modeloMo$;
\For{class en classList}
 	\State Combinamos el modelo $class$ con el $modeloMo$;
\EndFor

\For{ic en Conexión Interna del Modelo}
	\State (x,u)(y,j) $\gets$ ic
	\State u $\gets$ u + 1
	\Comment{ Se incrementa el orden de los puertos ya que en Modelica los arreglos comienzan en 1}
	\State j $\gets$ j + 1
  	\If{los modelos de ic son Escalares}
  		\State Se agrega la ecuación que representa la conexión entre el modelo x con el puerto u y el modelo y con el puerto j;
  	\ElsIf{los modelos de ic son Vectoriales}
  		\State Se agregan $N$ ecuaciones indexadas por $i$ que representan la conexión vectorial entre los modelos u y j mediante una sentencia \texttt{For};
  	\Else
  		\State No se conoce la conexión;
	\EndIf
\EndFor
\end{algorithmic}
 \caption{Pd2Mo::transform()}
\end{algorithm}

 
\section{Transformaciones para $\mu$-Modelica} \label{sec:transform}
	Tanto la clase \texttt{mda}, \texttt{prodint} e \texttt{If} son implementadas con el patrón de diseño de visitadores sobre 
	el árbol sintáctico abstracto\footnote{un árbol de sintaxis abstracta (AST), o simplemente un árbol de sintaxis, es una representación 
	de árbol de la estructura sintáctica abstracta (simplificada) del código fuente escrito en cierto lenguaje de programación.}, por lo que 
	cada clase es implementada heredando de una clase común (\texttt{Traverser}), la cual retorna una copia del AST y remplaza una parte de este según sea el 
	objetivo de la clase.
	
	Estas transformaciones son necesarias dado que al momento de realizar este trabajo algunas expresiones Modelica no eran soportadas por el QSS-Solver.
	 
	  \begin{itemize}
		\item  \texttt{mda}: Remplaza expresiones de la forma \texttt{X[N,k]}, donde $k \in \mathbb{N}$ o evalúa a una variable que evalúa a una expresión 
			$\in \mathbb{N}$. Son remplazados por \texttt{X\_k[N]} en las secciones \texttt{equation}, \texttt{algorithm}, \texttt{initial algorithm} y 
		declaraciones para hacer el código Modelica válido. En otras palabras remplaza arreglos bidimensionales por arreglos unidimensionales.
	
		En el Listado \ref{lst:preMda} se puede ver un arreglo bidimensional y en el Listado \ref{lst:posMda} el resultado de aplicar la transformación \texttt{mda}.

\begin{listing}[hcp]
\centering
\begin{minted}[breaklines=true]{modelica}
Real IndexShift_2_u[IndexShift_2_N,1];
\end{minted}
\caption{Ejemplo de arreglo bidimensional}\label{lst:preMda}
\end{listing}

\begin{listing}[hc]
\centering
\begin{minted}[breaklines=true]{modelica}
Real IndexShift_2_u_1[IndexShift_2_N];
\end{minted}
\caption{Ejemplo de arreglo unidimensional}\label{lst:posMda}
\end{listing}

		\item $prodint$: Remplaza expresiones de la forma $u[i, 1:nin] * w$ por expresiones de la forma 
			u[i,1] * w[1] + u[i,2] * w[2] .... + u[i,nin] * w[nin], donde $nin \in \mathbb{N}$ o evalúa a una variable que evalúa a una 
			expresión $\in \mathbb{N}$, es decir expande la operación de producto interno ($*$) entre los dos vectores.

		En el Listado \ref{lst:preProdInt}, se puede ver el producto interno entre dos vectores y en el Listado \ref{lst:posProdInt} el resultado de 
	la transformación $prodint$ cuando \texttt{VectorSum\_3\_nin = 4}  y \texttt{VectorSum\_3\_w} tiene dimensión $4$.

\begin{listing}[H]
\begin{minted}[breaklines=true]{modelica}
    VectorSum_3_y_1[VectorSum_3_i] = 
	VectorSum_3_u[VectorSum_3_i, 1:VectorSum_3_nin] * VectorSum_3_w;
\end{minted}
\caption{Ejemplo de producto Interno}\label{lst:preProdInt}
\end{listing}

	
\begin{listing}[H]
\begin{minted}[breaklines=true]{modelica}
    VectorSum_3_y[1,VectorSum_3_i] = VectorSum_3_u[1,VectorSum_3_i]*VectorSum_3_w[1]+ VectorSum_3_u[2,VectorSum_3_i]*VectorSum_3_w[2]+ VectorSum_3_u[3,VectorSum_3_i]*VectorSum_3_w[3]+ VectorSum_3_u[4,VectorSum_3_i]*VectorSum_3_w[4];
\end{minted}
\caption{Ejemplo de producto interno expandido}\label{lst:posProdInt}
\end{listing}

	\item $If$: Remplaza equaciones de la forma $if(v) \{eq_1\} else \{eq_2\}$ si $v$ evalúa a un booleano (a partir de parámetros o constantes, 
			es decir en análisis estático) se remplaza por $eq_1$ o $eq_2$ si es $v$ es verdadero o falso respectivamente.

	En el Listado \ref{lst:preIf} se puede ver un condicional antes de ser transformado y en el Listado \ref{lst:posIf} el resultado con \texttt{Shift $> 0$}).

\begin{listing}[htp]
\begin{minted}[breaklines=true]{modelica}
  if IndexShift_2_Shift > 0 then
    for IndexShift_2_i in 1:IndexShift_2_N-IndexShift_2_Shift loop
      IndexShift_2_y_1[IndexShift_2_i+IndexShift_2_Shift] = IndexShift_2_u_1[IndexShift_2_i];
    end for;
    for IndexShift_2_i in 1:IndexShift_2_Shift loop
      IndexShift_2_y_1[IndexShift_2_i] = 0;
    end for;
  else
    for IndexShift_2_i in 1:IndexShift_2_N-IndexShift_2_Shift loop
      IndexShift_2_y_1[IndexShift_2_i] = 
		IndexShift_2_u_1[IndexShift_2_i - +IndexShift_2_Shift];
    end for;
    for IndexShift_2_i in IndexShift_2_N + IndexShift_2_Shift : IndexShift_2_N loop
      IndexShift_2_y_1[IndexShift_2_i] = 0;
    end for;
  end if;
\end{minted}
\caption{Ejemplo de condicional}\label{lst:preIf}
\end{listing}


\begin{listing}
\begin{minted}[breaklines=true]{modelica}
  for IndexShift_2_i in 1:IndexShift_2_N-IndexShift_2_Shift loop
    IndexShift_2_y_1[IndexShift_2_i+IndexShift_2_Shift] = IndexShift_2_u_1[IndexShift_2_i];
  end for;
  for IndexShift_2_i in 1:IndexShift_2_Shift loop
    IndexShift_2_y_1[IndexShift_2_i] = 0;
  end for;
\end{minted}
\caption{Ejemplo de condicional expandido}\label{lst:posIf}
\end{listing}

\end{itemize}
