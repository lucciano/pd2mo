	A continuación se presenta el pseudo código que implementa las transformaciones descriptas en este trabajo, 
	el código completo puede encontrarse en \url{https://github.com/lucciano/pd2mo}, el cual utiliza dos librerias, Modelica C Compiler 
	\footnote{http://sourceforge.net/projects/modelicacc/} el cual nos permite manipular la estructura de los modelos y evaluar los parámetros.
y libreria de PowerDEVS \footnote{http://sourceforge.net/projects/powerdevs/} para leer los archivos PDS.

El programa esta separado en 4 módulos:

\section{Programa Principal}

El Programa principal en el archivo main.cpp, el cual es responsable de la interfaz con el usuario (linea de comando) y lanzar la transformación de la simulación, asi como establecer los archivos desde donde se lee y hacia donde se escriben la simulación de powerDEVS y Modelica, respectivamente.


\begin{function*}[H]
 modelCoupled *cm = parsePDS(QString::fromStdString(src\_infile))\;
 modelCoupled *qm = flatter::flat(cm)\;
 Pd2Mo q = Pd2Mo()\;
 q.transform(flatted, modelname, \&outfile, \&oFlogfile)\;
 AST\_StoredDefinition sd = parseFile(src\_outfile.c\_str(),\&amp;r)\;
 mda *m = new mda()\;
 prodint *  prod = new prodint()\;
 If *i = new If()\;
 
 outfile $\ll$ m$\rightarrow$visitClass( 
 	prod$\rightarrow$visitClass( 
 		i$
ightarrow$visitClass( 
 		\*sd$
ightarrow$models()$
ightarrow$begin()))) $\ll$ endl\;
 \caption{main(src\_infile)}
\end{function*}

\newpage

\section{Transformación Principal}
La clase \emph{Pd2Mo} implementa las principales partes de la transformación.

\begin{function}[H]
 modelCoupled *model = parsePDS(qfilename)\;
 AST\_ClassList classList = getAsClassList(model)\; 
 int j = 0\;
 \ForEach{class en classList}{
 	\If{La clase esta traducida a $\mu$Modelica}{
 		Prefijamos las variables con el nombre del modelo $class$ y la posición $j$ que ocupan en la lista\;
 		Remplazamos la entrada $class$ dentro de la lista por su copia producida en el paso anterior\;
 	}
 }
 Creamos un modelo $modeloMo$\;
 \ForEach{class en classList}{ 
 	Combinamos el modelo $class$ con el $modeloMo$\;
 }
 

 \ForEach{ic en Conexión Interna del Modelo}{
	\tcc{Las conexiones ic estan definidas como dos pares de números, cada par señalan número  de modelo y número de puerto, en este caso los puertos deben ser desfasados en uno, pues los puertos en nuestra representación son los sub-indices de $u$ e $y$ para cada modelo, pero el primer elemento de los arreglos en Modelica comienzan}
	\If{Si los modelos de ic son Escalares}{
		Se agrega la ecuación que representa la conexión entre los modelos\;
	}\ElseIf{Si los modelos de ic son Vectoriales}{
		Se agregan $N$ ecuaciones indexadas por $i$ que representa la conexión, vectorial entre los modelos mediante una sentencia For\;
	}\Else{
		No se conoce la conexión;
	}
 }
 \caption{Pd2Mo::transform()}
\end{function}

\section{Aplanado de modelos acoplados}
La clase \emph{flatter} implementa el aplanado de los modelos acoplados.

\begin{algorithm}[H]
 \ForEach{ModeloHijo en Lista de Modelos}{
 	\If{Tipo de ModeloHijo es COUPLED}{
 		\ForEach{ModeloHijo2 en Lista de ModeloHijo$
ightarrow$ModeloHijo}{
 			 	\If{Tipo de ModeloHijo2 es ATOMIC}{
 			 		Copiamos el ModeloHijo2 al ModeloResultado\;
 			 	}\Else{
 			 		Copiamos el aplanado de ModeloHijo2\;
 			 	}
 			 	\ForEach{Conexión del Modelo}{
 			 		\If{Si la conexión involucra un modelo \quotes{aun no procesado}}{
 			 			Las conexiones deben ser modificadas teniendo en cuenta los modelos agregados en el aplanado\;
 			 		}
 			 		\If{Si la conexión involucra como destino el modelo acoplado ModeloHijo}{
 			 			Se crea una nueva conexión (en ModeloResultado) entre los modelos agregado recientemente según la conexión del puerto de entrada del ModeloHijo y el origen de la conexión\;
 			 			La conexión se marca para ser borrada\;
 			 		}
 			 		\If{Si la conexión involucra como origen el modelo acoplado ModeloHijo}{
 			 			Se crea una nueva conexión (en ModeloResultado) entre los modelos agregado recientemente según la conexión del puerto de salida del ModeloHijo y el destino de la conexión\;
 			 			La conexión se marca para ser borrada
 			 		}
 			 		\If{Si la conexión fue marcada}{Se borra la conexión}
 			 	}
 		}
 		
 	}\Else{
 		Copiamos el nodo ModeloHijo al ModeloResultado \;
 		Copiamos las conexiones del ModeloHijo y cualquier otro ModeloHijo que ya haya sido procesado
 	}
 }
 return ModeloResultado
 \caption{flatter::flat}
\end{algorithm}

\section{Transformaciones para $\mu$Modelica}
La clase \emph{mda}, \emph{prodint}, \emph{If} implementan transformaciones enfocadas en $\mu$Modelica.

Tanto la clase $mda$, $prodint$ y $If$ recorren el árbol abstracto sintáctico (AST), por lo que cada clase es implementada heredando de una clase común (Traverser), la cual retorna una copia del AST y remplaza una parte este.

 \begin{itemize}
 	\item  $mda$: Remplaza expresiones de la forma $X[N,k]$, donde $k \in \mathbb{N}$ o evaluá a una variable que evaluá a una expresión $\in \mathbb{N}$, es remplazado por $X_k[N]$
 	
 	\item $prodint$: Remplaza expresiones de la forma $u[i, 1:nin] * w$ por expresiones de la forma u[i,1] * w[1] + u[i,2] * w[2] .... + u[i,nin] * w[nin], donde $nin \in \mathbb{N}$ o evaluá a una variable que evaluá a una expresión $\in \mathbb{N}$
 	
 	\item $If$: Remplaza expresiones de la forma $if(v){eq_1}else{eq_2}$ si $v$ evaluá a un booleano (a partir de parámetros o constantes, es decir en análisis estático) se remplaza por $eq_1$ o $eq_2$ si es $v$ es verdadero o falso respectivamente.
 \end{itemize}
