	A continuación se presenta el pseudo código que implementa las transformaciones descriptas en este trabajo, 
	el código completo puede encontrarse en \url{https://github.com/lucciano/pd2mo}, el cual utiliza dos librerias, Modelica C Compiler 
	\footnote{http://sourceforge.net/projects/modelicacc/} el cual nos permite manipular la estructura de los modelos y evaluar los parámetros.
	y libreria de PowerDEVS \footnote{http://sourceforge.net/projects/powerdevs/} para leer los archivos PDS.

El programa esta separado en 4 módulos:

\section{Programa Principal}

El Programa principal en el archivo main.cpp, el cual es responsable de la interfaz con el usuario (linea de comando) y lanzar la transformación de la simulación, asi como establecer los archivos desde donde se lee y hacia donde se escriben la simulación de powerDEVS y Modelica, respectivamente.


\begin{algorithm}[H]
\begin{algorithmic}[1]
\State modelCoupled *cm $\gets$ parsePDS(QString::fromStdString(src\_infile));
\State modelCoupled *qm $\gets$ flatter::flat(cm);
\State Pd2Mo q $\gets$ Pd2Mo();
\State q.transform(flatted, modelname, \&outfile, \&oFlogfile);
\State AST\_StoredDefinition sd $\gets$ parseFile(src\_outfile.c\_str(),\&amp;r);
\State mda *m $\gets$ new mda();
\State If *i $\gets$ new If();
\State outfile $\ll$ m$\rightarrow$\Call{visitClass} 
		{prod$\rightarrow$visitClass( i$\rightarrow$visitClass( 
			*sd$\rightarrow$models()$\rightarrow$begin()))} $\ll$ endl;

\end{algorithmic}
\caption{main(src\_infile)}
\end{algorithm}

\section{Transformación Principal}
La clase \emph{Pd2Mo} implementa las principales partes de la transformación, la cual incluye abrir el archivo PDS, e invocar el aplanado, obtener los diferentes modelos modelica que representan los modelos atómico, prevenir la colisión de nombres, crear el modelo final y realizar las conexiones.

\begin{algorithm}[H]
\begin{algorithmic}[1]
\State modelCoupled *model $\gets$ parsePDS(qfilename);
\State AST\_ClassList classList $\gets$ getAsClassList(model); 
\State int j $\gets$ 0\;
\For{class en classList}
 	\If{La clase esta traducida a $\mu$Modelica}

 		\State Prefijamos las variables con el nombre del modelo $class$ y la posición $j$ que ocupan en la lista;
 		\State Remplazamos la entrada $class$ dentro de la lista por su copia producida en el paso anterior;
 	\EndIf
\EndFor
\State Creamos un modelo $modeloMo$;
\For{class en classList}
 	\State Combinamos el modelo $class$ con el $modeloMo$;
\EndFor

\For{ic en Conexión Interna del Modelo}
	\State Las conexiones ic estan definidas como dos pares de números, cada par señalan número  de modelo y número de puerto, en este caso los puertos deben ser desfasados en uno, pues los puertos en nuestra representación son los sub-indices de $u$ e $y$ para cada modelo, pero el primer elemento de los arreglos en Modelica comienzan.
  	\If{los modelos de ic son Escalares}
  		\State Se agrega la ecuación que representa la conexión entre los modelos;
  	\ElsIf{los modelos de ic son Vectoriales}
  		\State Se agregan $N$ ecuaciones indexadas por $i$ que representa la conexión, vectorial entre los modelos mediante una sentencia \texttt{For};
  	\Else
  		\State No se conoce la conexión;
	\EndIf
\EndFor
\end{algorithmic}
 \caption{Pd2Mo::transform()}
\end{algorithm}

\section{Aplanado de modelos acoplados}
La clase \emph{flatter} implementa el aplanado de los modelos acoplados descripto en la sección \ref{aplanado}.
 
\begin{algorithm}[H]
\begin{algorithmic}[1]
\For{ModeloHijo en Lista de Modelos}
  	\If{Tipo de ModeloHijo es COUPLED}
  		\For{ModeloHijo2 en Lista de ModeloHijo$\rightarrow$ModeloHijo}
  			 	\If{Tipo de ModeloHijo2 es ATOMIC}
  			 		\State Copiamos el ModeloHijo2 al ModeloResultado;
				\Else
  			 		\State Copiamos el aplanado de ModeloHijo2;
				\EndIf
  			 	\For{Conexión del Modelo}
  			 		\If{Si la conexión involucra un modelo \quotes{aun no procesado}}
  			 			\State Las conexiones deben ser modificadas teniendo en cuenta los modelos agregados en el aplanado;
					\EndIf
  			 		\If{Si la conexión involucra como destino el modelo acoplado ModeloHijo}
  			 			\State Se crea una nueva conexión (en ModeloResultado) entre los modelos agregado recientemente según la conexión del puerto de entrada del ModeloHijo y el origen de la conexión;
  			 			\State La conexión se marca para ser borrada;
					\EndIf
  			 		\If{Si la conexión involucra como origen el modelo acoplado ModeloHijo}
  			 			\State Se crea una nueva conexión (en ModeloResultado) entre los modelos agregado recientemente según la conexión del puerto de salida del ModeloHijo y el destino de la conexión;
  			 			\State La conexión se marca para ser borrada;
					\EndIf
  			 		\If{Si la conexión fue marcada}
						
						\State Se borra la conexión
					\EndIf
				\EndFor
		\EndFor
  	\Else

  		\State Copiamos el nodo ModeloHijo al ModeloResultado 
  		\State Copiamos las conexiones del ModeloHijo y cualquier otro ModeloHijo que ya haya sido procesado

	\EndIf
\EndFor
\Return ModeloResultado
\end{algorithmic}
\caption{flatter::flat}
\end{algorithm}

 
\section{Transformaciones para $\mu$-Modelica} \label{sec:transform}
	Tanto la clase \texttt{mda}, \texttt{prodint} y \texttt{If} son implementadas con el patron de diseño de visitadores sobre 
	el árbol sintáctico abstracto\footnote{un árbol de sintaxis abstracta (AST), o simplemente un árbol de sintaxis, es una representación 
	de árbol de la estructura sintáctica abstracta (simplificada) del código fuente escrito en cierto lenguaje de programación.}, por lo que 
	cada clase es implementada heredando de una clase común (\texttt{Traverser}), la cual retorna una copia del AST y remplaza una parte este según sea el 
	objetivo de la clase.
	
	Estas transformaciones son necesarias dado que al momento de realizar este trabajo no son soportadas por el QSS-Solver, ya que no son $\mu$-Modelica
	valido.
	Arreglos multidimensionales no están soportados dado que no es $\mu$-Modelica valido, ya que los arreglos son de una sola dimensión, mientras que el producto 
	vectorial de dos vectores no estaba implementado en un principio, ya se encuentra implementado en una versión de desarrollo, y la transformación \texttt{If}
	fue implementada para simplificar le código final.
	 
	  \begin{itemize}
		\item  \texttt{mda}: Remplaza expresiones de la forma \texttt{X[N,k]}, donde $k \in \mathbb{N}$ o evalúa a una variable que evalúa a una expresión 
			$\in \mathbb{N}$, es remplazado por \texttt{X\_k[N]}.

\begin{figure}[htp]
\centering
\begin{cminted}{modelica}
Real IndexShift_2_u[IndexShift_2_N,1];
\end{cminted}

$\Downarrow$

\begin{cminted}{modelica}
Real IndexShift_2_u_1[IndexShift_2_N];
\end{cminted}
\end{figure}


		\item $prodint$: Remplaza expresiones de la forma $u[i, 1:nin] * w$ por expresiones de la forma 
			u[i,1] * w[1] + u[i,2] * w[2] .... + u[i,nin] * w[nin], donde $nin \in \mathbb{N}$ o evaluá a una variable que evaluá a una 
			expresión $\in \mathbb{N}$

\begin{figure}[htp]
\centering
\begin{minted}{modelica}
    VectorSum_3_y_1[VectorSum_3_i] = 
	VectorSum_3_u[VectorSum_3_i, 1:VectorSum_3_nin] * VectorSum_3_w;
\end{minted}

	(Donde \texttt{VectorSum\_3\_nin = 4}  y \texttt{VectorSum\_3\_w} tiene dimensión 4, lo que es necesario para que quede definida el producto de dos 
	vectores en modelica.)

$\Downarrow$

\begin{minted}{modelica}
    VectorSum_3_y[1,VectorSum_3_i] = 
	VectorSum_3_u[1,VectorSum_3_i]*VectorSum_3_w[1]+
	VectorSum_3_u[2,VectorSum_3_i]*VectorSum_3_w[2]+
	VectorSum_3_u[3,VectorSum_3_i]*VectorSum_3_w[3]+
	VectorSum_3_u[4,VectorSum_3_i]*VectorSum_3_w[4];
\end{minted}
\end{figure}

		\item $If$: Remplaza expresiones de la forma $if(v){eq_1}else{eq_2}$ si $v$ evaluá a un booleano (a partir de parámetros o constantes, 
			es decir en análisis estático) se remplaza por $eq_1$ o $eq_2$ si es $v$ es verdadero o falso respectivamente.

\begin{figure}[htp]
\centering
\begin{minted}{modelica}
  if IndexShift_2_Shift > 0 then
    for IndexShift_2_i in 1:IndexShift_2_N-IndexShift_2_Shift loop
      IndexShift_2_y_1[IndexShift_2_i+IndexShift_2_Shift] = 
	IndexShift_2_u_1[IndexShift_2_i];
    end for;
    for IndexShift_2_i in 1:IndexShift_2_Shift loop
      IndexShift_2_y_1[IndexShift_2_i] = 0;
    end for;
  else
    for IndexShift_2_i in 1:IndexShift_2_N-IndexShift_2_Shift loop
      IndexShift_2_y_1[IndexShift_2_i] = 
		IndexShift_2_u_1[IndexShift_2_i - +IndexShift_2_Shift];
    end for;
    for IndexShift_2_i in IndexShift_2_N + IndexShift_2_Shift : IndexShift_2_N loop
      IndexShift_2_y_1[IndexShift_2_i] = 0;
    end for;
  end if;
\end{minted}

(Con \texttt{Shift $> 0$})

$\Downarrow$

\begin{minted}{modelica}
  for IndexShift_2_i in 1:IndexShift_2_N-IndexShift_2_Shift loop
    IndexShift_2_y_1[IndexShift_2_i+IndexShift_2_Shift] = 
	IndexShift_2_u_1[IndexShift_2_i];
  end for;
  for IndexShift_2_i in 1:IndexShift_2_Shift loop
    IndexShift_2_y_1[IndexShift_2_i] = 0;
  end for;
\end{minted}
\end{figure}
	  \end{itemize}
